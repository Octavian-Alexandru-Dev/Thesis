\cleardoublepage\phantomsection\pdfbookmark{Sommario}{Sommario}
\begingroup
\let\clearpage\relax
\let\cleardoublepage\relax
\let\cleardoublepage\relax

\chapter*{Sommario}

Il presente documento descrive il lavoro svolto durante il periodo di tirocinio formativo, della durata di circa trecento ore complessive, dal laureando Samuele Calugi presso l'azienda Geckosoft, sede di Pisa.
Durante il tirocinio era richiesto il raggiungimento di massima dei seguenti obiettivi:
\begin{itemize}
    \item studio autonomo degli Standard OGC (Web Map Service, Web Feature Service, Web Coverage Service, Web Map Tile Service) e dei formati dati geospaziali (GeoJSON, Shapefile, ecc.);
    \item sviluppo di un frontend interattivo in grado di mostrare mappe ottenute attraverso gli standard OGC e i formati dati geospaziali specificati;
    \item sviluppo di un backend per il fornimento delle mappe in formato conforme agli standard OGC e ai formati dati geospaziali richiesti;
    \item integrazione del backend e del frontend per creare un'applicazione completa e funzionale per la visualizzazione di dati geospaziali provenienti da diverse fonti.
    \item l'utilizzo di \textit{Angular}, per la parte front-end e \textit{.NET} per la parte back-end;
    \item documentazione dettagliata del lavoro svolto, incluse le tecnologie utilizzate, le scelte progettuali e le modalità di utilizzo del prodotto.
\end{itemize}

\endgroup

\vfill